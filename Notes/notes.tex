\documentclass[11pt]{article}
\usepackage{fontspec}
\usepackage{unicode-math}
\usepackage{amsmath}

\setmainfont{texgyrepagella-regular.otf}[
  BoldFont = texgyrepagella-bold.otf,
  ItalicFont = texgyrepagella-italic.otf,
  BoldItalicFont = texgyrepagella-bolditalic.otf
]


\title{Notes on Causality}
\author{Tomás Paiva de Lira}

\begin{document}
\maketitle

\begin{abstract}
The subject of causality has accompanied men on the journey of \textit{understanding} since its inception: from Aristotle's postulation of the four causes that delineate physical phenomena, to Hume's framework of constant conjunctions.
A paradigm shift occurred at the turn of the 20th century, a time when the philosophy of Positivism held the most adherence in the scientific community. As a result, the founders of modern statistics, such as Karl Pearson and Francis Galton, rejected the inquiry into that which had no empirical grounding. Restricting the scope of statistics to answering "what" rather than "why".
However, motivated to mitigate the missing data problem, Rubin proposed a model based on potential outcomes that was inherently causal, but restricted its scope to measuring effects. Rubin's Causal Model joined causality back with statistics. 
The objective of these notes is to elucidate the modern applications of causality through the agnostic perspective that measure theory permits. Moreover, once proven, these causal techniques are put to the test with real examples to understand their limitations and successes.
 
\end{abstract}

\tableofcontents
\newpage

\begin{section}{Potential Outcomes}
Hello

\end{section}{Potential Outcomes}



\end{document}
